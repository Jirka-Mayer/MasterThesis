%%% Šablona pro jednoduchý soubor formátu PDF/A, jako treba samostatný abstrakt práce.

\documentclass[12pt]{report}

\usepackage[a4paper, hmargin=1in, vmargin=1in]{geometry}
\usepackage[a-2u]{pdfx}
\usepackage[czech]{babel}
\usepackage[utf8]{inputenc}
\usepackage[T1]{fontenc}
\usepackage{lmodern}
\usepackage{textcomp}

\begin{document}

%% Nezapomeňte upravit abstrakt.xmpdata.

Optické rozpoznávání notových zápisů je úzký podobor počítačového vidění, který sice disponuje určitým množstvím anotovaných datasetů, nicméně má k dispozici řádově větší množství neanotovaných dat. Tento obor se v poslední době vyvíjí zejména díky aplikaci hlubokého učení, ale na trénování neuronových sítí se zatím používají pouze anotovaná data. Semi-supervised learning je podoblast strojového učení, zbývající se současným učením z anotovaných a neanotovaných dat. Cílem je získat lepší modely, než kdybychom trénovali pouze z anotovaných dat. V této práci jsme upravili existující architekturu, používanou pro detekci hudebních symbolů, a navrhli jsme způsob, jakým ji trénovat v semi-supervised režimu. Upravená architektura je schopná učit se reprezentace i z neanotovaných dat a ve srovnání se svojí původní variantou má stabilnější trénování.

\end{document}
